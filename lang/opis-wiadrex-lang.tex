\documentclass{article}

\usepackage[T1]{fontenc}
\usepackage[utf8]{inputenc}
\usepackage[polish]{babel}

\newcommand\setItemnumber[1]{\setcounter{enumi}{\numexpr#1-1\relax}}

\title{opis wiadrex lang}
\author{Marcin Abramowicz \\ ma406058}
\date{April 2021}

\begin{document}

    \maketitle

    Wiadrex lang jest lekko ulepszoną wersją Latte (Latte++), z małymi zapożyczeniami ze Scali.
    
    Dokładniejszy opis punktów z tabelki:
    
    \begin{enumerate}
        \item
            Zawiera 3 typy: Int, Bool i String - jak w Latte.
            
        \item
            Arytmetyka i operacje jak w Latte.
        
        \item
            Tworzenie nowych zmiennych odbywa się przy użyciu operacji: `var <nazwa zmiennej>: <typ zmiennej> = <wartosc zmiennej>`.
            
        \item
            Wypisywanie odbywa się przy pomocy `printInt`, `printBool` oraz `printString`, odpowiednio dla typów.
            
        \item
            Bloki przy `while` i `if` / `if/else` *muszą* być otoczone nawiasami \{\} - brak nawiasów jest brzydki i utrudniający prace programisty, więc język to wymusza.
            
        \item
            Funkcje  tworzy się następująco: `fun <nazwa funkcji>(<lista argumentów odzielone przecinkiem>): <zwracany typ> {<cialo funkcji>}`
            
        \item 
            Argumenty funkcji są postaci: `<nazwa argumentu>: <typ argumentu>` taki argument jest przekazywany przez wartość. Aby przekazywać argumenty przez referencje trzeba dodać słówko kluczowe `var` przez zmienną: `var <nazwa argumentu>: <typ argumentu>`.
            
        \setItemnumber{11}
        \item
            Funkcje tak jak w opisie wyżej, mogą zwracać wartość danego typu.
            
        \setItemnumber{14}
        \item
            Można tworzyć rekordy (podobnie jak case class w Scali): `record <nazwa rekordu>(<pola rekordu>)`, gdzie pola rekordu są takie same jak argumenty funkcji. Nowe rekordy tworzy się przy użyciu operatora `new`: `new <nazwa rekordu>(<argumenty rekordu>)`.
        
        \setItemnumber{17}
        \item
            Zmienne mogą być typu funkcyjnego, maja wtedy typ: `(<lista typów argumentów>) -> <zwracany typ>`. Można tworzyć lambdy: `lambda <argumenty jak w funkcji> => {<ciało>}`
    \end{enumerate}
    
    Pozostałe zaznaczone punkty, które nie zostały tutaj opisane są standardowe.
    

\end{document}
